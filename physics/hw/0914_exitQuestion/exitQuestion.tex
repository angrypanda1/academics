\documentclass[a4paper]{report}

\usepackage[utf8]{inputenc}
\usepackage[T1]{fontenc}
\usepackage{textcomp}
\usepackage[english]{babel}
\usepackage{amsmath, amssymb}


% figure support
\usepackage{import}
\usepackage{xifthen}
\pdfminorversion=7
\usepackage{pdfpages}
\usepackage{transparent}
\newcommand{\incfig}[1]{%
    \def\svgwidth{\columnwidth}
    \import{./figures/}{#1.pdf_tex}
}

\pdfsuppresswarningpagegroup=1
\title{9.14 Exit Question}
\author{Svadrut Kukunooru}
\begin{document}
    \maketitle
\section{Question}%
\label{sec:Question}
An astronaut rotates at the end of a test machine whoe arm has a length of $10.0\ \text{m}$. If the maximum acceleration she reaches must not exceed $5g$, what is the maximum number of revolutions per minute of the arm? 
\section{Solving}%
\label{sec:Solving}
\subsection{Variables}%
\label{sub:Variables}
\[
    g = 9.81 \times 5 = 49.05
.\] 
\[
    L = 10.0\ m
.\] 
\subsection{Solve}%
\label{sub:Solve}
We can use the equation for acceleration, 
\[
    a_c = \frac{v^2}{r}
.\] 
Plugging values in, 
\[
49.05 = \frac{v^2}{10}
.\] 
\[
v = 22.14\ \frac{m}{s}
.\] 
Divide by the circumference and multiply by 60 to get the total period. 
\[
    \frac{22.14}{20\pi} \times 60 = \fbox{21.14\ \text{rpm}}
.\] 
\end{document}
