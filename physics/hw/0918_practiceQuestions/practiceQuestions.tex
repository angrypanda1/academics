\documentclass[a4paper]{report}

\usepackage[utf8]{inputenc}
\usepackage[T1]{fontenc}
\usepackage{textcomp}
\usepackage[english]{babel}
\usepackage{amsmath, amssymb}


% figure support
\usepackage{import}
\usepackage{xifthen}
\pdfminorversion=7
\usepackage{pdfpages}
\usepackage{transparent}
\newcommand{\incfig}[1]{%
    \def\svgwidth{\columnwidth}
    \import{./figures/}{#1.pdf_tex}
}

\pdfsuppresswarningpagegroup=1
\title{9.18 UCM Practice Questions}
\author{Svadrut Kukunooru}
\begin{document}
    \maketitle
    \begin{enumerate}
        \item An object hangs from a light string and moves in a horizontal circle of radius $r$. The string makes an angle $\theta$ with the vertical. The angular speed of the object is $\omega$. What is $\text{tan}\ \theta$? \fbox{A}
        \item An object of mass $m$ makes $n$ revolutions per second around a circle of radius $r$ at a constant speed. What is the kinetic energy of the object? \fbox{C}
        \item A motorcyclist is cornering on a curved race track. Which combination of change soft banking angle $\theta$ and coefficient of friction $\mu$ between the tyres and road allows the motorcyclist to travel around the corner at greater speed? \fbox{A}
        \item The mass at the end of a pendulum is made to move in a horizontal circle of radius $r$ at constant speed. The magnitude of the net force on the mass is $F$. What is the direction of $F$ and the work done by $F$ during half a revolution? \fbox{A}
        \item An object of mass $m$ at the end of a string of length $r$ moves in a vertical circle at a constant angular speed $\omega$. What is the tension in the string when the object is at the bottom of the circle? \fbox{A}
        \item A mass at the end of a string is swung in a horizontal circle at increasing speed until the string breaks. What is the subsequent path taken by the mass? \fbox{D}
        \item A horizontal disc rotates uniformly at a constant angular velocity about a central axis norma lto the plane of the disc. Point $X$ is a distance $2L$ from the centre of the disc. Point Y is a distance $L$ from the centre of the disc. Point Y has a linear speed $v$ and a centripetal acceleration $a$. \fbox{B}
        \item An electron moves with uniform circular motion in aregion of magnetic field. Which diagram shows the acceleration $a$ and velocity $v$ of the electron at point P? \fbox{B}
        \item An object rotates in a horizontal circle when acted on by a centripetal force $F$. What is the centripetal force acting on the ojbect when the radius of the cicle doubles and the kinetic energy of the object halves? \fbox{A}
        \item The maximum speed with which a car can take a circular turn of radius $R$ is $v$. The maximum speed with which the same car, under the same conditions, can take a circular turn of radius $2R$ is what? \fbox{B}
        \item A car on a road follows a horizontal circular path at constant speed. Which of the following correctly identifies the origin and the direction of the net force on the car? \fbox{D}
        \item Aibhe and Euan are sitting on opposite sides of a merry-go-round, which is rotating at constant speed around a fixed centre. The diagram below shows the diagram from above. Aibhe is moving at speed 1.00 $m \cdot s^{-1}$ relative to the ground. 
            \begin{enumerate}
                \item Determine the magnitude of the velocity of Aibhe relative to
                \begin{enumerate}
                \item Euan. \\ \\
                    \fbox{$0\  m \cdot s^{-1}$, since the distance between the two isn't changing.} \\
                \item the centre of the merry-go-round \\ \\
                    \fbox{$0\ m \cdot s^{-1}$, since the distance between the two isn't changing.} \\
                \end{enumerate}
                \item \textbf{Do 4 things.}
                    \begin{enumerate}
                        \item Outline why Aibhe is accelerating even though she is moving at constant speed. \\ \\ 
                        Even though the magnitude of her velocity isn't changing, her direction is. \\
                    \item Draw an arrow on the diagram on page 22 to show the direction in which Aibhe is accelerating. \\ \\ 
                        (I can't draw, so imagine an arrow from Aibhe to the center.) \\ 
                    \item Identify the force that is causing Aibhe to move in a circle. \\ \\ 
                        The centripetal force is pulling Aibhe towards the center. 
                    \item The diagram below shows a side view of Aibhe and Euan on the merry-go-round. Explain why Aibhe feels as if her upper body is being "thrown outwards", away from the center of the merry-go-round. \\ \\ 
                        Aibhe's upper body has inertia, unlike her lower body which is being pulled into the center of the circle. Therefore, she feels "pulled outwards". 
                    \end{enumerate}
                    \item Euan is rotating on a merry-go-round and drags his foot along the ground to act as a brake. The merry-go-round comes to a stop after 4.0 rotations. The radius of the merry-go-round is 1.5 m .The average frictional force between his foot and the ground is 45 N. Calculate the work done.  \\ \\ 
            You can find the distance traveled by Euan by calculating circumference:
            \[
                4.0 \times 2\pi \times 1.5 = 37.70\ \text{m}
            .\] 
            \[
                W = -Fd = 45 \times 37.70 = \fbox{1700\ J}
            .\]
        \item Aibhe moves so that she is sitting at a distance of 0.75 m from the centre of the merry-go-round, as shown below. Euan pushed the merry-go-round so that he is again moving at $1.0\ ms^{-1}$ relative to the ground. 
            \begin{enumerate}
                \item Determine Aibhe's speed relative to the ground. \\ \\ 
                    You can calculate velocity with the equation
                    \[
                    v = \frac{2\pi r}{T}
                    .\] 
                    Since $r$ is halved, that means that $v$ is also halved. Therefore, \\
                \begin{equation}                    
                    \boxed{
                    v = 0.5\ \frac{m}{s}
            .}
                \end{equation}
                \item Calculate the magnitude of Aibhe's acceleration. \\ \\ 
                    You can calculate acceleration with the equation
                    \[
                    a = \frac{v^2}{r}
                    .\] 
                    \[
                    a = \frac{0.5^2}{0.75}
                    .\] 
                    \begin{equation}
                        \boxed{a = 0.33\ \frac{m}{s}}
                    \end{equation}
            \end{enumerate}
            \end{enumerate} 
            \item This question is about circular motion. The diagram shows a car moving at a constant speed over a curved bridge. At the position shown, the top surface of the bridge has a radius of curvature of 50 m. 
            \begin{enumerate}
                \item Explain why the car is accelerating even though it is moving with a constant speed. \\ \\ 
                    \fbox{The car is accelerating because its direction is changing.} \\ 
                \item On the diagram, draw and label the vertical forces acting on the car in the position shown. \\ \\ 
                    \noindent\fbox{%
                    \parbox{\textwidth}{%
                (I can't draw, so imagine a longer arrow going down and two small normal forces going up from the wheels.)
            }%
            }
            \item Calculate the maximum speed at which the car will stay in contact with the bridge. 
                You can calculate maximum speed with the equation
                \begin{equation}
                    g = \frac{v^2}{r}
                \end{equation}
                \begin{equation}
                    v = \sqrt{50 \times 9.8}
                \end{equation}
                \begin{equation}
                    \boxed{22\ m \cdot s^{-1}.}
                \end{equation}
            \end{enumerate}
    \end{enumerate}
\end{document}
