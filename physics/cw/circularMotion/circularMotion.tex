\documentclass[a4paper]{report}

\usepackage[utf8]{inputenc}
\usepackage[T1]{fontenc}
\usepackage{textcomp}
\usepackage[english]{babel}
\usepackage{amsmath, amssymb}


% figure support
\usepackage{import}
\usepackage{xifthen}
\pdfminorversion=7
\usepackage{pdfpages}
\usepackage{transparent}
\newcommand{\incfig}[1]{%
    \def\svgwidth{\columnwidth}
    \import{./figures/}{#1.pdf_tex}
}

\pdfsuppresswarningpagegroup=1
\title{Circular Motion}
\author{Svadrut Kukunooru}
\begin{document}
    \maketitle
    \section{Circular Motion Intro}
    \begin{itemize}
        \item \textit{Angular Displacement} ($\theta$)
        \begin{itemize}
            \item Angle which an object moves through in an arc
            \item Measured in \textit{radians}
            \item \textit{NOT A VECTOR QUANTITY} (unlike linear displacement)
            \item $\theta = \frac{s}{r}$ is how you can relate arc length to radius, $r$, and distance traveled on the arc, $s$. 
        \end{itemize}
        \item \textit{Angular Speed} is calculated with the equation $\omega = \frac{\Delta \theta}{\Delta t}$. 
            \begin{itemize}
                \item Units: $\text{rad} \cdot s^{-1}$
            \end{itemize}
        \item \textit{Period (T)}
            \begin{itemize}
                \item The time required for an object to make one complete cycle
            \end{itemize}
        \item \textit{Frequency (f)}: The number of cycles per second (measured in \textbf{Hz})
        \item Frequency can be related to period with the equation
            \[
            f = \frac{1}{T}
            .\] 
    \end{itemize}
    \subsection{Practice Problems}
    \begin{enumerate}
        \item A rope goes over a circular pulley with a radius of 6.50 cm. If the pulley makes exactly 4 revolutions without the rope slipping, what length of rope passes over the pulley? \\ \\
            For this problem, all you need to do is calculate the circumference and multiply it by 4.
            \[
                C = \pi(13) \times 4 = \fbox{$52\pi$}
            .\] 
        \item A ball with a radius of 15.0 cm rolls on a level surface. The translational speed of the center of mass is 0.250 $m \cdot s^{-1}$.  What is the angular speed about the center of mass if the ball rolls without slipping? \\ \\ 
            Angular speed can be related to translational speed with the equation
            \[
            \omega = \frac{v_t}{r}
            .\] 
            Therefore, 
            \[
                \frac{2 \times 0.250}{0.15} = \fbox{$3.33\ \frac{rad}{s}$}
            .\] 
        \item A bocce ball with a diameter of 6.00 cm rolls without slipping on a level lawn. It has an initial angular speed of 2.35 $\frac{rad}{s}$ and comes to rest after 2.50 m. Assuming uniform acceleration (deceleration), determine the following:
            \begin{enumerate}
                \item The magnitude of its angular deceleration \\ \\
                    You can find angular deceleration with the kinematic equation
                    \[
                        \omega_f^2 = \omega_0^2 + 2\alpha(\Delta \theta)
                    .\] 
                    You can find angular displacement with the equation
                    \[
                    \theta = \frac{s}{\pi d}
                    .\] 
                    Plugging values in, you get
                    \[
                        \frac{2.50}{\pi(0.06)} = 13.26\ \text{rotations}
                    .\] 
                    Converting to radians gives you
                    \[
                        13.26 \times \frac{2\pi}{1\ \text{rotation}} = 83.6\ \text{rad}
                    .\] 
                    Plugging everything into the original equation:
                    \[
                        0^2 = (2.35)^2 + 2\alpha(83.6)
                    .\] 
                    \[
                        \fbox{$\alpha = -0.033\ \text{rad} \cdot s^{-2}$}
                    .\] 
                    \item The magnitude of its lineaer deceleration \\ \\ 
                        With the equation 
                         \[
                        v= \omega r
                        .\] 
                        \[
                        v = 2.35 \times 0.0300
                        .\] 
                        \[
                        u = \fbox{$0.0705\ m \cdot s^{-1}$}
                        .\] 
            \end{enumerate}
    \end{enumerate}
    \section{Uniform Circular Motion Post-Lab Discussion}
    \begin{itemize}
        \item Radius's effect on force is \textbf{decreasing exponential}
        \item Mass's effect on force is \textbf{increasing linear}
        \item Speed's effect on force is \textbf{increasing exponential}
    \end{itemize}
    From these conclusions, you can calculate centripetal force with the equation
    \[
    \Sigma F_c = \frac{mv^2}{r}
    .\] 
    \subsection{Practice Questions}
    What is the average speed of the Earth in its orbit around the Sun given the information below?
    \[
        m_{earth} = 5.98 \times  10^{24}  \\ 
    \] 
    \[
    r_orbit = 1.50 \times 10^{11} 
\]
\[    
    F_{g(E+S)} = 3.53 \times 10^{22}
    .\] 
    You can calculate the speed of the earth with the equation 
    \[
    v = \sqrt{\frac{F_c r}{m}} 
    .\] 
    \[
        v = \sqrt{\frac{3.53 \times 10^{22} \times 1.50 \times 10^{11}}{5.98 \times 10^{24}}} 
    .\] 
    \[
        \fbox{$v = 2.98 \times 10^4$}
    .\] 
    \subsection{Lecture, Continued}
    Since Newton's 2nd law is 
    \[
        F_{net} = ma
    .\] 
    Therefore, the centripetal acceleration can be calculated by 
    \[
    a_c = \frac{v^2}{r}
    .\] 
    \subsection{Sample Problem}
    A stunt pilot in an air show performs a loop-the-loop in a vertical circle of radius $3.21 \times 10^3$ m. During this performance the pilot whose wight is 806 N, maintains a constant speed of $2.25 \times 10^2$ m/s. 
    \begin{enumerate}
        \item What is the magnitude of the centripetal force acting on the pilot? \\
        You can calculate centripetal force with the equation
        \[
        F_c = \frac{mv^2}{r}
        .\] 
        The variables we have are $m = \frac{806}{9.8}\ \text{kg}$, $v = 2.25 \times 10^2\ \frac{m}{s}$, and radius $3.21 \times 10^3\ \text{m}$. Therefore, when we substitute all these values in, we get 
        \[
            F_c = \frac{82.24 \times 50625}{3.21 \times 10^3} = 1297\ N
        .\] 
        \item What is the magnitude of the normal force acting on the pilot at the top of his path? 
        We can calculate this with the equation
        \[
        F_n = F_c - F_g = \frac{mv^2}{r} - mg = 1297 - 806 = 491\ N
        .\] 
        \item What is the magnitude of the normal force acting on the pilot at the bottom of the path?
            \[
            F_N = F_c + F_g = \frac{mv^2}{r} + mg = 1296 + 806 = 2100\ N
            .\] 
        \item What is the minimum velocity the pirate can travel and still maintain circular motion? \\
        At the top, if you go at minimum speed, the weight is the only thing that is causing circular motion. Therefore,
        \[
        \frac{v^2}{3.21\times 10^3} = 9.8
        .\] 
        \[
        v = 177\ \frac{m}{s}
        .\] 
    \end{enumerate}
    \subsection{Sample Problem}
    Keys, $m = 150.0\ g$, are attached to a string and swung around in a vertical circle with radius of $35.0 cm$.
    \begin{enumerate}
        \item What is the minimum speed at which the keys must be moving in order to remain in uniform circular motion? \\
        \[
        \frac{v^2}{0.35} = 9.8
        .\] 
        \[
        v = 1.85\ \frac{m}{s}
        .\] 
        \item What is the tension in the string when the keys, at this speed, are at the bottom of their circular path? \\
        \[
        T = \frac{mv^2}{r} + mg
        .\] 
        \[
            T = \frac{0.15 \times 1.85^2}{0.35} + (0.15)(9.81)
        .\] 
        \[
        T = 2.94\ N
        .\] 
    \end{enumerate}

\end{document}
