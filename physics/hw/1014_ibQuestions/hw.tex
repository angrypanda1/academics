\documentclass[a4paper]{article}

\usepackage[utf8]{inputenc}
\usepackage{amsmath, amssymb}

\title{IB Questions}
\author{Svadrut Kukunooru}
\date{\today}

\begin{document}
    \maketitle
    \begin{enumerate}
        \item The gravitational potential is $V$ at a distance $R$ above the surface of a spherical planet of radius $R$ and uniform density. What is the gravitational potential a distance $2R$ above the surface of the planet? \\ \\
        You calculate gravitational potential with the equation
            \[
            -\frac{GM}{r}
            .\] 
        Because the radius increases from $2R$ to $3R$, the answer is \fbox{D}.
    \item The escape velocity for an object at the surface of the Earth is $v_{esc}$. The diameter of the Moon is 4 times smaller than that of the Earth and the mass of the Moon is 81 times smaller than that of the Earth. What is the escape velocity of the object on the Moon? \\ \\ 
        \[
        v = \sqrt{\frac{2GM}{R}} 
        .\] 
    Therefore, the answer is $\frac{2}{9}v_{esc}$, or \fbox{C}. 
        \item A satellite in a circular orbit around the Earth needs to reduce its orbital radius. What is the work done by the satellite rocket engine and the change in kinetic energy resulting from this shift in orbital height. \\ \\ 
        Since both potential and kinetic energy increase, the work done is also positive. Therefore, the answer is \fbox{A}. 
    \item Satellite X is in orbit around the Earth. An identical satellite Y is in a higher orbit. What is correct for the total energy and the kinetic energy of the satellite Y compared with satellite X? \fbox{A}
    \item The diagram shows 5 gravitational equipotential lines. The gravitational potential on each line is indicated. A point mass $m$ is placed on the middle line and is then released. \fbox{A}
    \item The mass of the Earth is $M_E$ and the mass of the Moon is $M_M$. Their respective radii are $R_E$ and $R_M$. Which is the ratio \[
            \frac{\text{escape speed from the Earth}}{\text{escape speed from the Moon}}
    ?\]
    \fbox{C}
\item The sketch graph shows how the gravitational potential $V$ of a planet varies with distance $r$ from the centre of the planet of radius $R_0$. \fbox{A}
    \end{enumerate} 
\end{document}
