\documentclass[11pt, letterpaper]{article}
\usepackage{fancyhdr}
\title{How Have Religious Policies Changed in India, from the Mughals to the Present Day?}\label{religious-discrimination-in-india}
\author{Svadrut Kukunooru}
\date{\today}
\begin{document}
\begin{titlepage}
    \maketitle
\end{titlepage}
\tableofcontents
\newpage
\section{Abstract}\label{abstract}

On December 11, 2019, the Citizenship Amendment Bill passed the higher
house of Indian Parliament 125-105. The bill amends the 64-year-old
Indian citizenship law and gives certain religions -- that is, ``Hindu,
Sikh, Buddhist, Jain, Parsi, and Christian'' a much better chance for
obtaining citizenship than Muslims \cite{BBC}. This bill has been met with
widespread protests that the bill ignores the Indian constitution's
secular nature and ``explicitly and blatantly seeks to enshrine
religious discrimination into law'' \cite{BBC}. In other words, the law
seeks to deem Muslims as illegal migrants and attempt to deport them out
of India.

This is not the only modern-day instance of discrimination against
Muslims in India; according to a survey of Indian police adequacy, ``One
in two police personnel feels that Muslims are likely to be''naturally
prone" towards committing crimes." \cite{police}. 

However, this religious discrimination is not getting much traction in
the American -- or, for that matter, Western -- press, which is why I
have chosen to investigate this issue in my response to bring more focus
to this important conflict.

The question I am looking to talk about in my essay is

\begin{quote}
How have religious policies changed in India, from the Mughals to the
present day?
\end{quote}

In order to answer the question with as much efficiency as possible, I
will split the quasi-modern history of India into three periods: the
Mughal Empire, the British Raj, and post-independence India. Each of
these periods will be divided into two categories: how religion was
beneficial in politics during the time period, and how it was not. In
this way, it is easier to see which side has more evidence overall, and
a more genuine conclusion can be drawn based on the evidence gathered.

\section{The Mughal Empire}\label{the-mughal-empire}

Mughal policy towards Hindus waxed and waned during the empire's time in
power, according to the preferences of its current emperor. The emperors
I will discuss are a selection of ``The Great Mughals'' -- Babur, Akbar,
Shah Jahan, and Aurangzeb. Out of the nineteen emperors who ruled over
India, I have selected these four based on their effect on the rise (and
decline) of the Mughal Empire and their somewhat colorful religious
views.

Babur was the first emperor of the Mughal empire, the direct descendant
of Genghis Khan and Tamerlaine. Under his rule, Hinduism was tolerated
and new temples were even allowed to be built. According to the BBC,
``The Empire {[}Babur{]} founded was a sophisticated civilization based
on religious toleration. It was a mixture of Persian, Mongol and Indian
culture.'' \cite{babur}. In fact, ``His first act after conquering Delhi was to
forbid the killing of cows because that was offensive to Hindus.''
\cite{babur}. Babur still declared wars against Indian kings like Rana Sanga
and Medini Rai as \emph{jihads} (holy wars), but this was for political
rather than religious reasons -- he just wanted to inspire his followers
to fight for him in the name of religion. Dr.~S.R. Sharma writes,
``There is no evidence of {[}Babur{]} ever having destroyed a Hindu
temple or otherwise persecuted the Hindus on account of their
religion.'' \cite{sharma}.

Babur's grandson, Akbar, was known as the most religiously tolerant
emperor of the Mughal Empire. Akbar's leniency towards religion was
probably due to several factors -- his Sunni father and Shi'a mother,
the liberal religious views of his tutor Abdul Latif, and most of all
the preaching of groups like the Bhaktis and Sufis, who prioritized
religious peace over conflict. A similar religious Reformation was
taking place in India during Akbar's rule, and Akbar made sure that the
same ideas were present in his reign. Early in his reign, Akbar
abolished numerous policies that discriminated against non-Muslims
religions, erasing the slave-trade in 1562, pilgrim tax in 1563, and the
\emph{jizya} tax in 1564. He constructed Ibadat Khana, or a house of
worship, where scholars from all major religions (Hindus, Parsis, Jains,
Christians) had neutral discussions with one another. Christians were
allowed to establish churches in places like Cambay, Lahore, Hugli, and
Agra. He allowed state positions to be open to people of all religions
whereas before it was only limited to people of the Muslim faith. Akbar
went so far in his acceptance of religion that he made his own religion
that drew from the teachings of multiple religions (mostly Islam,
Hinduism, and Zoroastrianism) , called the Din-i-Ilahi. It prioritized a
yearning for god (like Sufism, celibacy as a virtue (Catholicism), and
condemned the killing of animals (Jainism). Even though the maximum
amount of followers it had hovered at around 20 people, Akbar's attempt
to make a religion that best served his goal of ruling a predominantly
Hindu state while being Muslim illustrates his tolerance of religion.

According to most accounts, Islam's ``popularity'' in India reached
its peak under the rule of emperor Aurangzeb of the Mughal Empire, who
created the world's biggest economy at the time and made India produce
25\% of the world's industrial output until 1750 \cite{partha}. To
better understand how a foreign Islam-preferring empire improved India
to the point of being one of the world's major powers at the time, we
have to take a look at their law documents; most importantly, the
\emph{Fatawa 'Alamgiri}, a comprehensive guide made during the height of
the Mughal Empire that cited Islamic law in regulations in ``statecraft,
general ethics, military strategy, economic policy, justice and
punishment'' \cite{malik}. This document is held in great renown for being
one of the best examples of Islamic law in action. To create it,
Aurangzeb gathered 500 people who were experts in Islamic law (from
India, Iraq, and Saudi Arabia).(Unfortunately, there are no English
translations of the full work, so I will use Neil B.E. Baillie's
\emph{Digest of Muhammadan Law}, which was used by British courts in
India as a reference to Islamic law and largely based on chosen excerpts
from the \emph{Fatawa 'Alamgiri}.)

A pattern can be seen in most of these laws in that it creates ``a legal
system that treats people differently based on their religion, social
class, and economic status.'' \cite{baillie}. Most of these laws favor Muslims
in criminal law, pillage and slavery, and virtually forced non-Muslim
people, or \emph{dhimmi} living under Aurangzeb's rule to convert to
Islam. For example, only Muslims could own property and slaves. The
\emph{dhimmi} still had certain rights; according to Sharia law, the
state had the obligation to protect the individual's ``life, property,
and freedom of religion'' \cite{glenn}. However, they had to pay for this
with loyalty to the state and the payment of a \emph{jizya} tax, which
Aurangzeb reimposed on the Hindus. (originally, Akbar had eliminated
this tax, which originally applied to People of the Book (e.g.~Jews and
Christians) but its scope was later expanded to include Hindus as well).

However, saying that Aurangzeb was an anti-Hindu fanatic would be a huge
exaggeration; we have to consider the politics at the time that caused
him to pass these laws in the middle of his reign. In fact, at the
beginning of his reign, when asked why he was hiring so many Hindus to
fill the ranks of the bureaucracy, Aurangzeb responded, ``What
connection have earthly affairs with religion? And what right have
administrative works to meddle with bigotry? 'For you is your religion
and for me is mine.'' \cite{baillie}. However, popular
opposition against how Aurangzeb took power (killed his brothers and
imprisoned his father, Shah Jahan) prevented him from claiming that he
was a protector of \emph{sharia}. In addition, numerous revolts by the
Rajputs and Marathas as well as the Mughal expansion into the Deccan
prompted Aurangzeb to have harsher policies towards Hindus.

With these laws, Aurangzeb doomed the empire into a period of decline;
the ``lesser'' Mughal emperors after Aurangzeb became more invested in
maintaining their lavish lifestyle with harsh taxes than actually
governing. As a result, The Mughal Empire gradually lost large tracts of
land to the Maratha Empire in central India, and soon after the Maratha
Empire became the ``protectors'' of the Mughal Emperor in Delhi. This
arrangement continued until the British East India Company defeated the
Marathas in the Third Anglo-Maratha War, exiled the last Mughal Emperor
Bahadur Shah Zafar, and began rule over India, only to give their
territory over to the British empire after the Indian Rebellion of 1857.
A new period in the history of India was about to begin.

\section{The British Raj}\label{the-british-raj}

After the Indian rebellion, the ruling class of the British Empire tried
to reflect on what had caused the rebellion. As a result, they concluded
that their social reforms were one of the major factors that caused the
rebellion, such as the ban on \emph{sati} (the practice of widows'
ritual suicide after their husband died) and the reforms against
remarriage of Hindu child widows. This mentality can be seen in Queen
Victoria's proclamation, where she declares that ``We {[}the British
Empire{]} disclaim alike our Right and Desire to impose Our Convictions
on any of Our Subjects..'' and ``We hold ourselves bound to the natives
of our Indian territories by the same obligation of duty which bind us
to all our other subjects.'' \cite{act1858}.

The primary motive behind Britain's unique religious policy in India was
to make ruling easier. As there were only 20,000 British soldiers ruling
over 300 million Indian civilians, they required much more than raw
manpower and guns to keep the huge population in check.

The British used two methods in terms of religion to attempt to keep the
population in check:

\begin{itemize}
\item
  Promoting Christianity in India
\item
  When that didn't work, pitting the Hindu/Muslim populations against
  each other
\end{itemize}

It is also important to note that while Britain had plenty of experience
fighting (and sometimes trading) with Muslims through the Crusades and
other holy wars, the only exposure they had to Hindus was through a
dubious fork in the Silk Route -- China was the only major Asian country
that Britain had interacted with up until the 17th century.

\subsection{Christianity}\label{christianity}

The Prime Minister Robert Gascoyne-Cecil best states the mentality that
the British had on Christianity in India; ``It is not only our duty but
is in our interest to promote the diffusion of Christianity as far as
possible throughout the length and breadth of India.'' \cite{kanjamala} Promoting
Christianity was a pattern seen in almost all of the British Empire's
territories; in fact, according to Jean Comaroff, Christianity was seen
as colonialism's ``agent, scribe, and moral alibi.'' \cite{comaroff}. When
used as a weapon, Christianity divided already fragmented Hindu-Muslim
populations into a conflicted mass that became much easier to govern.

However, this mentality backfired on the British. In fact,
missionary-built schools created a collaborative space to debate on the
relationship between Christianity and Hinduism (the topic of Islam was
present at these debates, but Hinduism was the dominant topic of
discussion). When missionaries attempted to ``make forced connections
between {[}the two religions{]}'', such as likening the concept of Hindu
duality to Judeo-Christian theology or finding similarities between the
afterlives of Hinduism and Judaism, students, \emph{pandits}, and
\emph{sannyasis} responded by engaging in theological debates with
missionaries and attempting to find a middle ground between the two
religions.

What the British more so ended up doing was educating thousands of
Indians at missionary-built schools, arguing that ``western scholarship
was saturated with Christian morals and that such ethoses would
transform Indians accordingly'' (Bellenoit). Their attempts to use the
schools as a means to convert religious susceptible children to
Christianity usually backfired on them; Reverend Worthington Jones noted
that ``when there is a chance of {[}Indian parents'{]} boys becoming
Christian they {[}withdraw their children from the school{]}''. The
debate mentality also spread to the validity of the curriculum being
taught at these schools as well; Europeans realized that a religion
``which absorbed animism was not likely to strain at swallowing
science.'' (Bellenoit). In fact, Western schools of thought were thought
to have a constructive rather than destructive effect on Indian
knowledge systems; one student believed that the religious and physical
world existed on two different planes and that he accepted both schools
of thought.

Indians usually came to missionary-built schools ``not for spiritual
inquiry, but for the practicalities of education, for which `much credit
is due to the missionaries.'\,'' (Bellenoit).

\subsection{Hindu-Muslim
Fragmentation}\label{hindu-muslim-fragmentation}

After the Indian Revolt of 1857, where Hindus and Muslims fought side by
side against the East India Company, The British had realized that ``the
widespread unity of feeling that had developed among the Sepoys of the
Bengal army was responsible for their defalcation. Differences of
{[}religion{]} had been rubbed away by contact in the ranks, and the
army had become as one. It was therefore natural for it to act together
and, under the pressure of the multitude of factors that caused the
Indian Mutiny, it revolted.'' \cite{mason} . To avoid this ever happening
again, the British instituted several policies in the structure of their
native Indian Army to keep different groups or religions distrustful of
one another, such as decentralizing their Indian army to keep local
feuds and tensions between religious groups running high. Brigadier John
Coke states this mentality best: ``Our endeavor should be to uphold in
full force the separation which exists between the different religions
and races, and not to endeavor to amalgamate them. \emph{Divide et
impera} should be the principle of Indian Government.'' \cite{guilford}

As a response to these harsh British policies, British Secretary to the
Department of Revenue, Agriculture, and Commerce Allan Hume wrote a
letter to graduates of Calcutta University requesting them to form their
own political movement. The first meeting of the INC was held in Bombay
in 1885, and although originally a group serving the interests of the
British-educated Indian elite, it gradually transformed into a
anti-Muslim nationalist movement advocating the independence of India
from Britain (and Muslims) after the First Partition of Bengal in 1905.
The British had never intended to separate Hindus and Muslims in the
Partition of Bengal, but accidentally ended up doing just that and their
repealing of the partition was interpreted by Muslims as a British
compromise to appease the Hindu majority while disregarding Muslim
interests.

However, this shift to Hindu nationalism alienated many Muslim members
of the Indian National Congress, causing them to demand separate
electorates for Muslim representation in the Imperial Council. The
British, recognizing that supporting the Muslims in this venture would
further allow them to divide the Indian population into more easily
governable groups, acknowledged their ``representative character'' and
promoted separate Muslim electorates with the signing of the informal
Lucknow Pact, which separated India into two distinct religion-oriented
ruling groups. Eventually, this mentality solidified into the Two-Nation
Theory, which established the nationality for India Muslims as their
religion, not their ethnicity or language. This caused several Hindu
nationalist organizations to form with the same cause, citing safety for
people following different religions as the main reason for partition.

On April 13th, 1919, British soldiers killed 379 people, mostly Sikhs,
in what is now termed the Amritsar Massacre. Historians now agree that
this was most likely the catalyst for Britain's eventual withdrawal from
India and India's independence. As an immediate result, however, it
propelled Mohandas -- now Mahatma -- Gandhi into the limelight and
allowed him to showcase his non-cooperation campaign to the rest of
India. Gandhi restructured Congress, allowing membership to all castes,
and began leading even more extensive civil disobedience campaigns such
as the Salt Satyagraha, where thousands of Indians marched to the coast
and made their own salt in protest for Britain's oppressive salt tax.

The final straw was Britain declaring war on the Axis Powers on behalf
of India without consulting the (Hindu-majority) Congress, causing many
ministries to resign to protest the unjust declaration. Contrastingly,
the Muslim League held celebrations for the declaration and declared
their support for Britain. A year later, Jinnah held a speech in the
Congress saying that ``Muslims and Hindus\ldots were irreconcilably
opposed monolithic religious communities and as such, no settlement
could be imposed that did not satisfy the aspirations of the former.'' \cite{talbot}
Additionally, the Congress's increasingly extreme requests for
independence reform caused the nervous British to jail them until August
1945, allowing the ranks of the Muslim League to swell exponentially.
After a last-gasp attempt by the British to persuade the Muslim League
to support an independent India failing with the Cabinet Mission plan
and riots occurring between Hindu and Muslim groups causing more than
4000 deaths, the British empire concluded that partition was the only
way to ensure a peaceful and efficient transition of power. However, the
announcing of the Radcliffe Line, the border between India and Pakistan,
caused a horrific amount of violence, the traces of which are still
apparent today. Historians Ian Talbot and Gurharpal Singh wrote:

\begin{quote}
There are numerous eyewitness accounts of the maiming and mutilation of
victims. The catalogue of horrors includes the disembowelling of
pregnant women, the slamming of babies' heads against brick walls, the
cutting off of the victim's limbs and genitalia, and the displaying of
heads and corpses. While previous communal riots had been deadly, the
scale and level of brutality during the partition massacres was
unprecedented. Although some scholars question the use of the term
`{[}genocide{]}' concerning the partition massacres, much of the
violence was manifested with genocidal tendencies. It was designed to
    cleanse an existing generation and prevent its future reproduction. \cite{talbot} 
\end{quote}

\section{Modern India}\label{modern-india}

Modern-day politics in India has always held traces of the horrors of
British-caused partition and the discriminatory policies of the Mughals
before them. As a result, Indian politics have taken on a victim
mentality based on whoever is ruling at the time, with them lashing out
against the opposite group.

However, this was not what the government of India wanted initially. In
the constitution of India, which went into effect on January 26, 1950,
India declared herself a nation with ``LIBERTY of thought, expression,
belief, faith and worship.''. The first Prime Minister of India,
Jawaharlal Nehru, echoed this need for freedom of religion in his
presidential address to the Lahore Congress in 1929; Nehru admitted that
he was ``hostile to the caste system\ldots{[}and{]} openly criticized
communalism and strongly denounced Hindu communal groups.'' \cite{oxford}  Nehru was most known for Article 44 of the Indian
Constitution, which states that ``The State shall endeavor to secure for
the citizens a uniform civil code throughout the territory of India.'' \cite{oxford} 
Although Nehru has been criticized for allowing Muslims to keep their
own Shari'a regarding marriage and inheritance, one must realize that
Nehru had to make concessions in order to keep the peace between Hindu
and Muslim groups and avoid a repeat of Partition violence.

The next prime minister, Indira Gandhi, was even more secular than her
predecessor, going so far in her policies that she was labeled a
socialist and kicked out of her own party. She was most known for her
foreign policy and less so for her domestic policy. The only
religious-based events worth mentioning during her time in power is
Operation Blue Star, which caused Gandhi's assassination. In June 1984,
Gandhi ordered Indian troops to enter the Golden Temple to remove a Sikh
terrorist group from the complex. In the fighting, ``hundreds to many
thousands'' \cite{hamlyn} of Sikh soldiers and innocent pilgrims were killed,
prompting two Sikh bodyguards to kill Gandhi in October 1984 and
anti-Sikh riots to sweep the nation, causing 3,000 people, mostly Sikhs,
to be killed.

The assassination of two prime ministers in a row prompted a new party
to come to power on the Indian stage -- the BJP, or Bharatiya Janata
Sangh. The BJP drastically differed from the majority party in that most
of its members were Hindu nationalist, advocating for an anti-Muslim and
all-Hindu India. Many unfortunate events transpired because of this
hard-line mentality, such as the riots that occurred when the BJP
attempted to build a temple dedicated to the Lord Rama at the ruins of
the Babri Mosque, built by Mughal Emperor Babur. Finally, the Gujarat
Riots in 2002, caused by the burning of a train carrying Hindu pilgrims,
caused thousands of Muslims to be killed in Gujarat and is said to have
a ``high level of state complicity.''.

\section{Conclusion}\label{conclusion}

To conclude, I would like to take a look at how Narendra Modi -- India's
current prime minister -- stands in terms of religion and religious
policies. In 2002, the burning of a train and death of 58 Hindu pilgrims
prompted a large-scale Hindu-Muslim riot in Gujarat that killed 1044 and
injured 2500. Modi, presiding over the Chief Ministerial office of
Gujarat during this time, was accused of causing and tolerating the
violence, as well as several government officials and police who gave
rioters the addresses of Muslims and Muslim-owned properties. Although
he was cleared, suspicions of him being a Hindu nationalist continued
until his prime ministerial run and beyond. In 2014, Modi's run for
prime minister downplayed his Hindu nationalist tendencies and
emphasized his desire for a ``united'', anti-corrupt India. After he was
elected, however, Modi's policies took a more nationalist,
Hindu-preferring tone. Recently, Modi has appointed Yogi Adityanath, a
``militant Hindu monk'' \cite{nyt} as the Chief Minister of Uttar
Pradesh, India's largest state. And recently, the Citizenship Amendment
Act has solidified discrimination against Muslims by allowing followers
of every religion except Islam to gain citizenship faster. This
religious discrimination will most likely continue to become more
extreme if left unchecked. The only way to stop this is to bring more
awareness to this issue -- in the media, in education, and in
discussion. I hope that with this paper, I have brought at least a
little more light to this issue.
\\ \\ \\ 
\textbf{Word Count}: 3615
\newpage
\begin{thebibliography}{100}
\bibitem{BBC} “Citizenship Amendment Bill: India’s New ‘anti-Muslim’ Law Explained.” BBC News, December 11, 2019, sec. India. https://www.bbc.com/news/world-asia-india-50670393.

\bibitem{police} “Status of Policing in India.” Centre for the Study Developing Societies, 2019. https://www.tatatrusts.org/upload/pdf/state-of-policing-in-india-report-2019.pdf.

\bibitem{babur} “BBC - Religions - Islam: Mughal Empire (1500s, 1600s).” Accessed February 16, 2021. https://www.bbc.co.uk/religion/religions/islam/history/mughalempire.

\bibitem{sharma} Sharma, S. R. Mughal Empire in India: A Systematic Study Including Source Material. Atlantic Publishers & Dist, 1999.

\bibitem{partha} Parthasarathi, Prasannan (2011). Why Europe Grew Rich and Asia Did Not: Global Economic Divergence, 1600-1850. p. 42.

\bibitem{malik} Jamal Malik (2008), Islam in South Asia: A Short History, Brill Academic, ISBN 978-9004168596, pp. 194-197

\bibitem{baillie} Baillie, N. A Digest of Muhammadan Law. Kazi Publications, Incorporated, 1989.

\bibitem{glenn}  Glenn, H. Patrick (2007). Legal Traditions of the World. Oxford University Press. pp. 218–219.

\bibitem{act1858} Office, Public Record. “Learning Curve British Empire.” Text;image. Public Record Office, The National Archives. Accessed February 16, 2021. https://www.nationalarchives.gov.uk/education/empire/transcript/g2cs4s4ts.htm.

\bibitem{comaroff} Comaroff, Jean; Comaroff, John (2010) [1997]. "Africa Observed: Discourses of the Imperial Imagination". In Grinker, Roy R.; Lubkemann, Stephen C.; Steiner, Christopher B. (eds.). Perspectives on Africa: A Reader in Culture, History and Representation (2nd ed.). Oxford: Blackwell Publishing.

\bibitem{kanjamala} Kanjamala, Augustine (21 August 2014). The Future of Christian Mission in India. Wipf and Stock Publishers. pp. 117–119. ISBN 9781620323151.

\bibitem{mason} A Matter of Honour – an Account of the Indian Army, its Officers and Men, Philip Mason, ISBN 0-333-41837-9, p. 261.

\bibitem{guilford} Stewart, Neil. "Divide and Rule: British Policy in Indian History." Science & Society 15, no. 1 (1951): 49-57. Accessed February 16, 2021. http://www.jstor.org/stable/40400043.

\bibitem{talbot} Talbot, Ian; Singh, Gurharpal (2009), The Partition of India, Cambridge University Press, ISBN 978-0-521-85661-4

\bibitem{oxford} Nanda, B. R. Nehru and Religion. Oxford University Press. Accessed February 16, 2021. https://oxford.universitypressscholarship.com/view/10.1093/acprof:oso/9780195645866.001.0001/acprof-9780195645866-chapter-6?q=culture.

\bibitem{hamlyn} Hamlyn, Michael (12 June 1984). "Amritsar witness puts death toll at 1000". The Times. p. 7.

\bibitem{nyt} Gettleman, Jeffrey, and Maria Abi-Habib. “In India, Modi’s Policies Have Lit a Fuse.” The New York Times, March 1, 2020, sec. World. https://www.nytimes.com/2020/03/01/world/asia/india-modi-hindus.html.
\end{thebibliography}
\end{document}
