\documentclass{article}
\usepackage[utf8]{inputenc}
\usepackage{amsmath, amssymb}
\usepackage{berenis}
\usepackage[LY1]{fontenc}
\usepackage{anyfontsize}
\title{Reading \& Response}
\author{Svadrut Kukunooru}


\begin{document}
    \maketitle
    NOTE: For my response, I chose the first article, \textit{Variety's} "Art" Review. 
\section{Response}%
\label{sec:Response}
This review of \textit{Art} by Greg Evans for \textit{Variety} offers a disinterested take on the play, with his final opinion of \textit{Art} being, "'Art', too thin to rank even as a minor masterpiece, nonetheless takes a deserved place in the Broadway gallery." As support for this opinion, Evans cites that fact that the entire play hinges on something so inconsequential that it could come from \textit{Seinfeld} (the funny thing is, I could imagine Jerry buying a painting and George getting irrationally angry). While I thought that the play explored modernist ideas that I'd never thought about seriously before, Evans describes the ideas in the play like so: "Profound? Not particularly. Revelatory? Hardly. The arguments over modernism seem so dated they’re quaint.." As this reviewer probably is (1) older than me and (2) has watched more plays, he has a different opinion of the play. This led me to consider what audience Reza was aiming for this play; did she want to warn older people about the dangers of pseudo-intellectualism, or teach younger people about the limits of friendship? 
\end{document}
