\documentclass[a4paper]{report}

\usepackage[utf8]{inputenc}
\usepackage[T1]{fontenc}
\usepackage{textcomp}
\usepackage[english]{babel}
\usepackage{amsmath, amssymb}


% figure support
\usepackage{import}
\usepackage{xifthen}
\pdfminorversion=7
\usepackage{pdfpages}
\usepackage{transparent}
\newcommand{\incfig}[1]{%
    \def\svgwidth{\columnwidth}
    \import{./figures/}{#1.pdf_tex}
}

\pdfsuppresswarningpagegroup=1
\title{Circular Motion: Textbook Notes}
\begin{document}
  \maketitle 
The \textbf{force} that provides the centripetal force is calculated with the equation 
  \[
  F = \frac{mv^2}{r} = m \omega ^2 r
  .\] 
  Also note that $\omega$ is angular velocity and can be related to frequency with the equation
  \[
  \omega = 2 \pi f
  .\] 
  \textbf{Angular Acceleration}, or the rate of change of ANGULAR speed with time, is calculated with the equation
  \[
  \alpha = \frac{\Delta \omega}{\Delta t}
  .\]
  Also note that angular accelerations is measured in $\text{rad} \cdot s^{-2}$.
\\ \\ 
Rotational motion also has 4 similar kinematic equations to normal motion:
\[
\omega_f = \omega_i + \alpha t
.\] 
\[
\theta = \omega_i t + \frac{1}{2}\alpha t^2
.\] 
\[
\omega_f^2 = \omega_i^2 + 2\alpha \theta
.\] 
\[
    \theta = \frac{(\omega_i + \omega_f)t}{2}
.\] 
Pretty self-explanatory. 
\end{document}

