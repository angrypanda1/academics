\documentclass[a4paper]{report}

\usepackage[utf8]{inputenc}
\usepackage[T1]{fontenc}
\usepackage{textcomp}
\usepackage[english]{babel}
\usepackage{amsmath, amssymb}


% figure support
\usepackage{import}
\usepackage{xifthen}
\pdfminorversion=7
\usepackage{pdfpages}
\usepackage{transparent}
\newcommand{\incfig}[1]{%
    \def\svgwidth{\columnwidth}
    \import{./figures/}{#1.pdf_tex}
}

\pdfsuppresswarningpagegroup=1
\title{Uniform Circular Motion}
\author{Svadrut Kukunooru}
\begin{document}
    \maketitle
    \begin{enumerate}
        \item What is the speed, in m/s, of the roller coaster at the top of the loop if the radius of curvature there is $18.0\ m$ and the downward acceleration of the char is $1.50g$? \\ \\
             \[
            a_c = \frac{mv^2}{r}
            .\] 
            \[
            14.7 = \frac{v^2}{18}
            .\] 
            \begin{equation}
                \boxed{v = 16.3\ \frac{m}{s}}
            \end{equation} \\ 
        \item A stunt pilot in an air show performs a loop-the-loop in a vertical circle of radius $3.18\times 10^3\ m$. During this performance the pilot, whose weight is $685\ N$, maintains a constant speed of  $2.10\times 10^2\ m$.
            \begin{enumerate}
                \item When the pilot is at the highest point of the loop, determine his apparent weight. \\ \\ 
                You can calculate the apparent weight at the top of the loop with the equation
                \[
                \frac{mv^2}{r} - mg
                .\] 
                Plugging all the answers in, you get \fbox{284 N}. 
                \item At what speed will the pilot experience weightlessness? \\ \\ 
                You can get this speed with the equation
                \[
                mg = \frac{mv^2}{r}
                .\] 
                Plugging all the values in, you get \fbox{177 m/s}. 
                \item When the pilot is at the lowest point of the loop determine his apparent weight. \\ \\ 
                \[
                mg + \frac{mv^2}{r} = N
                .\] 
                \begin{equation}
                    \boxed{1654\ \text{N}}.
                \end{equation}
                    
            \end{enumerate}
            \item An unbanked cure of radius 150 m is rated for a maximum speed of $35.5\ \frac{m}{s}$. At what maximum speed, in m/s, should a flat curve with radius $61.0$ m be rated?
            \[
            g = \frac{v^2}{r}
            .\] 
            We find $g$ to be 8.4 $\frac{m}{s^2}$. Plugging the values for the next curve, we find the maximum speed to be \fbox{22.6 m/s}. 
    \end{enumerate}
\end{document}
