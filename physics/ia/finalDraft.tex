\documentclass{article}
\title{FRISBEES AND FLIGHT
DISTANCE}
\author{Svadrut Kukunooru}
\date{February 26, 2021}

\begin{document}

\begin{titlepage}
    \maketitle
\end{titlepage}

The research question I have decided on is:

\begin{quote}
What is the effect of frisbee circumference on flight distance, assuming
you throw a frisbee at the optimal 10-20 degree angle?
\end{quote}

I've always wondered how these frisbees with a hole in the middle of
them fly so much farther than normal, conventional frisbees. I want to
investigate whether it's the circumference of the frisbee (since these
frisbees are usually bigger than the normal ones) or something else to
do with lift.

Finding an answer to the relationship between circumference of an object
and the aerodynamic capabilities of said object could have hundreds of
uses -- not just in aerospace fields, but also perhaps to protect
offshore drilling plants from huge waves.

\subsubsection{Hypothesis}\label{hypothesis}

The circumference of the frisbee will relate to the distance the frisbee
travels parabolically, with an optimal circumference for maximum
difference and the distance decreasing exponentially as the
circumference becomes bigger and smaller than the optimal value.

\section{METHODOLOGY}\label{methodology}

Once again, my hypothesis is:

\emph{The radius of the frisbee will relate to the distance the frisbee
travels parabolically, with an optimal radius for maximum difference and
the distance decreasing exponentially as the circumference becomes
bigger and smaller than the optimal value.}

My manipulated variable will be the radius of the frisbee, which will be
measured in centimeters with a ruler. For my range of values, I will be
using 12 centimeters to 16 centimeters, with 0.5 cm intervals, for a
total of 9 manipulations. I will do each of these manipulations for 5
trials. I picked these values due to the radius for the Frisbee that I
have used being 12 cm. It is nearly impossible for me to reduce the
radius of the frisbee using power tools accurately, so I will just
measure the distance the frisbee goes if the radius increases. I expect
a decline in distance if the radius gets bigger, since there has to be a
reason that all Frisbees are usually the same size. . I am using an
architect's ruler, which has a uncertainty of \(\pm \ 0.025\)
centimeters, which will be adequate for my measurement. My validity
measure is if the responding variable varies dramatically from the
expected value in the trial.

My responding variable is the distance that the frisbee flies, measured
in meters. I will be measuring it using a meter stick, which has an
uncertainty of \(\pm \ 0.1\) centimeters. To help keep the responding
variable safe, I will use a stand for the frisbee to keep the same angle
throughout all the trials and use the same pushing motion with my hand
to attempt to keep the speed the same. I recognize that the speed of my
hand might affect the Frisbee's distance, and will account for
variations in hand speed with uncertainty.

My controlled variables are:

\begin{itemize}
\tightlist
\item
  Angle that the frisbee flies; if the angle differs drastically, the
  distance the frisbee flies differs equally as drastically, since a
  Frisbee's flying distance is dependent on the amount of air lift
  beneath it, which is changed by the angle. I will try to keep the
  angle the same by using a ramp to launch the frisbee from. Since the
  ramp is unlikely to change angles in the space of a few trials, I will
  measure the angle of the ramp every four trials with a protractor.
\item
  The wind speed; if the wind differs drastically, the distance the
  frisbee flies will differ equally as drastically, since a Frisbee's
  flying distance is dependent on the amount of air lift beneath it,
  which is changed by the wind speed. I will try to keep the wind speed
  the same by conducting the experiment on a calm day, and waiting until
  I can feel no wind to start a trial.
\item
  The hand speed; if the hand speed differs drastically, the distance
  the frisbee flies will differ equally as drastically, since a
  Frisbee's flying distance is dependent on the speed at which it is
  launched. I will try to keep the speed the frisbee is launched at the
  same by using the same hand motion, and attempting to keep the speed
  the same. I have no way to verify that I am using close to the exact
  same speed for each trial, so I will have to eyeball it.
\end{itemize}

\subsection{MATERIALS}\label{materials}

\begin{itemize}
\tightlist
\item
  A frisbee with radius 12 cm.
\item
  A ruler to check the radius of the frisbee
\item
  A meter stick to check the distance of the frisbee
\item
  A wooden ramp with an incline of approximately 15 degrees, measured
  with a protractor
\item
  Paper, scissors, and tape
\item
  An assistant to measure where the frisbee touches down
\end{itemize}

\subsection{PROCEDURES}\label{procedures}

\begin{enumerate}
\def\labelenumi{\arabic{enumi}.}
\tightlist
\item
  You will need a wide open space for this experiment. Set the frisbee
  at the bottom of the ramp and clear space up for the trajectory of the
  frisbee.
\item
  With one smooth motion, push the frisbee off the ramp, putting your
  hand on the top of the frisbee. Make sure your assistant is ready to
  mark down the exact place where the frisbee touches the ground.
\item
  Measure the distance the frisbee travels from the ramp to the place it
  touches down with the meter stick and record the value.
\item
  Repeat steps 2 and 3 four more times.
\item
  Use scissors to cut out a hollow circle with that has a border radius
  of 0.5 inches. Tape this to the outside of the frisbee.
\item
  Repeat steps 2-4.
\item
  Repeat steps 5 and 6 7 more times, each time adding a circle to the
  outside of the frisbee.
\end{enumerate}

\subsection{SAFETY}\label{safety}

As you are throwing objects,you are recommended to wear goggles while
doing this experiment. Additionally, makes sure that the area around the
Frisbee's flight path is clear of people and objects.

\section{DATA COLLECTION}\label{data-collection}

\subsection{Table I}\label{table-i}

\begin{figure}
\centering
\includegraphics{/home/svadrut/.config/Typora/typora-user-images/image-20210220203303873.png}
\caption{image-20210220203303873}
\end{figure}

The Effect of Radius of Frisbee on Distance

\subsection{Table II}\label{table-ii}

\begin{figure}
\centering
\includegraphics{/home/svadrut/.config/Typora/typora-user-images/image-20210220201336461.png}
\caption{image-20210220201336461}
\end{figure}

Average Effect of Radius of Frisbee on Distance

\subsubsection{Sample Calculation for Average
Distance}\label{sample-calculation-for-average-distance}

The sample we will use is the average distance the frisbee traveled when
the frisbee radius was 13.00 centimeters.

\begin{enumerate}
\def\labelenumi{\arabic{enumi}.}
\item
  Identify any outliers.

  From this sample size, I can see that 3.67 is an outlier in this
  sample, so I will not consider that value while calculating the
  average.
\item
  Calculate the average. \[
  \frac{328+334+326+330}{4} = \fbox{337}
  \]
\end{enumerate}

\subsection{Graph I}\label{graph-i}

\begin{figure}
\centering
\includegraphics{/home/svadrut/Downloads/image.png}
\caption{image}
\end{figure}

Frisbee Distance vs.~Frisbee Radius

I used a quadratic curve of best fit since it seemed that a linear fit
would not be adequate for measuring the correlation between distance and
radius. My quadratic curve of best fit is \[
y = -20.43x^2+583.1x-3796
\] However, on closer inspection, it appears that the graph is divided
into two parts. The points from 12 cm to 14 cm are arrayed in an
increasing logarithmic curve, whereas the points from 14 cm to 15.5 cm
are arrayed decreasing linearly.

\begin{figure}
\centering
\includegraphics{/home/svadrut/Downloads/circles.png}
\caption{circles}
\end{figure}

Therefore, it is most optimal to simply linearize the points from 12 cm
to 14 cm and leave the rest of the points be.

\subsection{Table III}\label{table-iii}

\begin{figure}
\centering
\includegraphics{/home/svadrut/.config/Typora/typora-user-images/image-20210224171134709.png}
\caption{image-20210224171134709}
\end{figure}

The Effect of Radius of Frisbee on Distance²

\hypertarget{sample-calculation-for-distanceuxb2}{%
\subsubsection{Sample Calculation for
Distance²}\label{sample-calculation-for-distanceuxb2}}

\begin{enumerate}
\def\labelenumi{\arabic{enumi}.}
\item
  Square the distance. \[
  262^2 = \fbox{68644}
  \]
\item
  Calculate the relative uncertainty by adding the relative
  uncertainties together and then calculate the real uncertainty by
  multiplying the calculated relative uncertainty by the square of the
  distance. \[
  \left[ \left[ \frac{0.1}{262} \right]\cdot 100 \right]^2 = 0.06\ \%
  \]

  \[
  0.06\% \ \cdot 68644 = \fbox{41.186}
  \]
\end{enumerate}

\hypertarget{graph-ii}{%
\subsection{Graph II}\label{graph-ii}}

\begin{figure}
\centering
\includegraphics{/home/svadrut/.config/Typora/typora-user-images/image-20210225202427463.png}
\caption{image-20210225202427463}
\end{figure}

\begin{figure}
\centering
\includegraphics{/home/svadrut/.config/Typora/typora-user-images/image-20210225202518585.png}
\caption{image-20210225202518585}
\end{figure}

\begin{figure}
\centering
\includegraphics{/home/svadrut/.config/Typora/typora-user-images/image-20210225202659173.png}
\caption{image-20210225202659173}
\end{figure}

Frisbee Distance² vs Frisbee Radius less than 14 cm

The line of best fit for these 5 points that I calculated was \[
y = (3.392 \times10^4)x - (3.349 \times10^5)
\] The slope has an uncertainty of about \(1.0665 \times 10^4\). I
calculated this by calculating the difference between the maximum and
minimum slopes and dividing by 2.

However, we should take into account the other 3 points on the graph.

\subsection{Graph III}\label{graph-iii}

\begin{figure}
\centering
\includegraphics{/home/svadrut/Downloads/image (4).png}
\caption{image (4)}
\end{figure}

\begin{figure}
\centering
\includegraphics{/home/svadrut/Downloads/image (3).png}
\caption{image (3)}
\end{figure}

\begin{figure}
\centering
\includegraphics{/home/svadrut/Downloads/image (5).png}
\caption{image (5)}
\end{figure}

Frisbee Distance vs Frisbee Radius greater than 14 cm

The line of best fit for these 3 points that I calculated was \[
y = -17x+603
\]

I calculated the uncertainty as 0.12, calculating the difference between
the minimum and maximum slope and dividing by 2.

\subsection{CONCLUSION}\label{conclusion}

Therefore, in conclusion, the optimal radius of a frisbee for maximum
distance is about 14 inches. Before 14 inches, the distance the frisbee
travels approximately follows a positive logarithmic curve as the radius
increases. After 14 inches, the distance the frisbee travels decreases
by approximately 17 centimeters as the radius increases 1 centimeter.

My hypothesis was half right -- the frisbee does follow a negative
parabolic arc in terms of distance vs radius until 14 centimeters, but
then decreases linearly rather than exponentially after 14 centimeters.

Possible improvements I could make to the procedure include:

\begin{itemize}
\tightlist
\item
  Making a way to keep the speed at which the frisbee launches
  consistent, like a pulley or motor that moves at a constant speed that
  can be controlled
\item
  Performing the experiment in a completely controlled environment with
  no wind to affect the frisbee's flight distance
\item
  Having more manipulations of the data -- e.g.~more radii
\end{itemize}
\end{document}
